\documentclass[a4paper,10pt]{article}

%A Few Useful Packages
\usepackage{marvosym}
\usepackage{fontspec} 					%for loading fonts
\usepackage{xunicode,xltxtra,url,parskip} 	%other packages for formatting
\RequirePackage{color,graphicx}
\usepackage[usenames,dvipsnames]{xcolor}
\usepackage[big]{layaureo} 				%better formatting of the A4 page
% an alternative to Layaureo can be ** \usepackage{fullpage} **
\usepackage{supertabular} 				%for Grades
\usepackage{titlesec}					%custom \section
\usepackage{array}
%Setup hyperref package, and colours for links
\usepackage{hyperref}
\definecolor{linkcolour}{rgb}{0,0.2,0.6}
\hypersetup{colorlinks,breaklinks,urlcolor=linkcolour, linkcolor=linkcolour}

%FONTS
\defaultfontfeatures{Mapping=tex-text}
%\setmainfont[SmallCapsFont = Fontin SmallCaps]{Fontin}
%%% modified for Karol Kozioł for ShareLaTeX use
\setmainfont[
SmallCapsFont = Fontin-SmallCaps.otf,
BoldFont = Fontin-Bold.otf,
ItalicFont = Fontin-Italic.otf
]
{Fontin.otf}
%%%%gla

%CV Sections inspired by: 
%http://stefano.italians.nl/archives/26
%\titleformat{\section}{\Large\scshape\raggedright}{}{0em}{}[\titlerule]
\titleformat{\section}{\Large\scshape\raggedright}{}{0em}{}%[\titlerule]
\titlespacing{\section}{0pt}{15pt}{5pt}
%Tweak a bit the top margin
%\addtolength{\voffset}{-1.3cm}

%Italian hyphenation for the word: ''corporations''
\hyphenation{im-pre-se}

%-------------WATERMARK TEST [**not part of a CV**]---------------
\usepackage[absolute]{textpos}

\setlength{\TPHorizModule}{30mm}
\setlength{\TPVertModule}{\TPHorizModule}
\textblockorigin{2mm}{0.65\paperheight}
\setlength{\parindent}{0pt}

%--------------------BEGIN DOCUMENT----------------------
\begin{document}

%WATERMARK TEST [**not part of a CV**]---------------
%\font\wm=''Baskerville:color=787878'' at 8pt
%\font\wmweb=''Baskerville:color=FF1493'' at 8pt
%{\wm 
%	\begin{textblock}{1}(0,0)
%		\rotatebox{-90}{\parbox{500mm}{
%			Typeset by Alessandro Plasmati with \XeTeX\  \today\ for 
%			{\wmweb \href{http://www.aleplasmati.comuv.com}{aleplasmati.comuv.com}}
%		}
%	}
%	\end{textblock}
%}

\pagestyle{empty} % non-numbered pages

\font\fb=''[cmr10]'' %for use with \LaTeX command

%--------------------TITLE-------------
\par{\centering
		{\Huge Atul Singh \textsc{Arora}
	}\bigskip\par}

%--------------------SECTIONS-----------------------------------
%Section: Personal Data
\section{Personal}

\begin{tabular}{rl}
    %\textsc{Place and Date of Birth:} & Someplace, Italy  | dd Month 1912 \\
    \textsc{Address:}   & Rue Aviateur Thieffry 48, Bruxelles -- 1040 \\
    \textsc{Phone:}     & +32 2 650 29 72\\ 
    \textsc{Mobile:}      & +32 471 56 00 81\\
    \textsc{Internet:}     & \href{mailto:atul.singh.arora@ulb.ac.be}{atul.singh.arora@ulb.ac.be}, \href{https://atulsingharora.github.io}{atulsingharora.github.io}
\end{tabular}

%Section: Work Experience at the top
\section{Research}
\begin{tabular}{r|p{11cm}}
 \emph{Current}     & PhD Thesis, \textsc{Université libre de Bruxelles (ULB)}, Belgium \\
                    &\emph{Quantum Cryptography and Communication} \\
                    &\small Advisor: Prof. Jérémie \textsc{Roland}\\
 &\footnotesize{  Recently solved an interesting problem about quantum weak coin flipping, a cryptographic primitive.
                  Started studying for the next project, quantum communication complexity.
                  On the side, have been investigating foundational aspects of quantum mechanics.}\\
                    &\small{A.S.A., S. Weis, J. Roland. \href{https://arxiv.org/abs/1811.02984}{arXiv:1811.02984} (Accepted. QIP; Submitted to STOC) }\\
                    &\small{K. Bharti, A.S.A, L. C. Kwek, J. Roland. \href{https://arxiv.org/abs/1811.05294}{arXiv:1811.05294} (Submitted to PRL) }\\
 \multicolumn{2}{c}{} \\

 \textsc{2015-16} & Master's Thesis, \textsc{Indian Institute of Science Education and Research (IISER), Mohali}, India \\
                  &\emph{Contextuality in a Deterministic Quantum Theory}\\
                  &\small Advisor: Prof. Arvind\\ 
                  &\footnotesize{Concluded that contextuality is not a necessary feature of quantum mechanics and proposed an alternative, non functional-conistency, bolstered by an explicit construction. }\\
                  &\small{A.S.A., K. Bharti, Arvind. \href{https://arxiv.org/abs/1607.03498}{arXiv:1607.03498}; \href{https://doi.org/10.1016/j.physleta.2018.11.049}{Physics Letters A. (Nov 2018)}\\            
\multicolumn{2}{c}{} \\

\textsc{Summer}   & Intern, \textsc{University of Siegen}, Germany\\
2015                  & \emph{Towards a macroscopic test of local realism}\\
                      & \small Advisor: Prof. Otfried \textsc{Gühne}\\
                      & \footnotesize{Constructed a Bell inequality using observables bounded in phase space to probe local realism using macroscopic variables.} \\
                      & \small{A.S.A., A. Asadian. \href{https://arxiv.org/abs/1508.04588}{arXiv:1508.04588}; \href{http://dx.doi.org/10.1103/PhysRevA.92.062107}{Phys. Rev. A 92, 061207}\\

\multicolumn{2}{c}{} \\

\textsc{2011-14}  & Intern, {\footnotesize worked on yarn defect recognition using OpenCV at the} \textsc{Indian Institute of Technology (IIT), Bombay} 
                      {\footnotesize under the supervision of} \small{Prof Anirban Guha}; 
                      {\footnotesize worked on setting up an experiment to study the dynamics of a dipole lattice at the}
                      \textsc{National Physical Laboratory (NPL)}, New Delhi, India
                      {\footnotesize under the supervision of} \small{Dr Ravi Mehrotra};
                      {\footnotesize worked on quantum simulation at} \textsc{IISER Mohali}, India  
                      {\footnotesize under the supervision of} \small{Prof Arvind}; 
\end{tabular}

%Section: Education
\section{Education}
\begin{tabular}{rp{11cm}}	
  \emph{present} & Doctorat en Sciences de l'ingénieur et technologie,\\
 \textsc{Oct} 2016  & \textbf{Université libre de Bruxelles (ULB)}, Belgium.\\
                    & \\
 \textsc{July} 2016 & Bachelor and Master of Science with \textsc{Physics} major,\\
 \textsc{July} 2011 & \textbf{Indian Institute of Science Education and Research (IISER), Mohali}, India.\\
& CPI: \textbf{9.4} /10. \small Graduated with \emph{rank two.} \hyperlink{grades}{\hfill | \footnotesize Details at the end}\\&\\ 
% & Thesis: ``Sublinear and Locally Sublinear Prices'' | \small Advisor: Prof. Erio \textsc{Castagnoli}\\
% &\normalsize \textsc{Gpa}: 28.61/30\hyperlink{grds}{\hfill | \footnotesize Detailed List of Exams}\\&\\
\end{tabular}

%Section: Scholarships and additional info
\section{Recognition }
\begin{tabular}{rp{11cm}}
 \textsc{Dec} 2016 & Two year research fellowship from the Belgian \emph{FNRS (Fonds National Recherche de Science)}, through the FRIA grant.\normalsize\\
 2016     & Top 5\% in the physics stream of the \emph{Graduate Aptitude Test in Engineering (GATE)}, India. \\
          & Obtained a 92.3 percentile in the national graduate physics exam, \emph{Joint Entrance Screening Test (JEST)}, India. \\
% \end{tabular}

% \begin{tabular}{rp{11cm}}
 2015     & Awarded the \emph{Junior Research Fellowship (JRF-NET)} from the Council of Scientific and Industrial Research, India. \\
          & Awarded the \emph{DAAD WISE} fellowship for a summer internship by and in Germany.\\
 2013-16  & Awarded the Certificate of Merit for the best academic performance in a semester, twice by IISER. Was among the highest scorers four other times.\\
 2012     & Awarded the \emph{KVPY} fellowship for my work on Stepper Motor Control, by DST, India.\\
 2010     & Granted financial support for attending the Bright Green Youth climate summit, Denmark.
\end{tabular}

%Section: Languages
\section{Languages}
\begin{tabular}{rl}
 \textsc{Hindi:}&Fluent\\
\textsc{English:}&Fluent\\
\textsc{Punjabi:}&Intermediate\\
\textsc{French:}&Basic\\
\end{tabular}

\section{Teaching}
\begin{tabular}{rrp{11cm}}
& ~~2019 &(scheduled) Teaching Assistant. Information Quantique (graduate). ULB, Brussels.\\  
& ~~2016 &Teaching Assistant. Thermodynamics (undergraduate). IISER, Mohali.\\
& ~~2015 &Teaching Assistant. Classical Mechanics (undergraduate). IISER, Mohali.
\end{tabular}

\section{Conferences}
\begin{tabular}{rrp{11cm}}
  & ~~2018 &\textsc{QuantAlgo} Workshop, Université Paris-Diderot, Paris. Talk. \emph{Quantum Weak Coin Flipping beyond bias 1/6}. \\  
  & ~~2018 &\textsc{QIP}, TU Delft. Poster. \emph{Quantum Weak Coin Flipping with bias 1/10}. \\
  & ~~2017 &\textsc{TQC}, Paris. Attended.
  \end{tabular}

% \section{Computer Skills}
% \begin{tabular}{rl}
%  Basic Knowledge:& \textsc{php}, my\textsc{sql}, \textsc{html}, Access, \textsc{Linux}, ubuntu, {\fb \LaTeX}\setmainfont[SmallCapsFont=Fontin-SmallCaps.otf]{Fontin.otf}\\
% Intermediate Knowledge:& \textsc{vba}, Excel, Word, PowerPoint\\
% \end{tabular}

\section{Interests \& Extracurricular}
Technology, Open-Source, Programming;\\ 
Philosophy---Ethics, Books---non-fiction; \\
Fitness; Piano, Guitar.
% \section{Extracurricular}

\textsc{\href{https://quic-meets.github.io}{QuIC-meets}} --- \emph{Organizer.} {\footnotesize A fortnightly department meet to facilitate collaboration. Status: Pilot. } \\
\textsc{\href{https://c-est-ca.github.io}{C'est ça}} --- \emph{Editor.} {\footnotesize An at least bilingual peer-reviewed popular science journal. Status: Initialising. }\\
\textsc{\href{https://gleaned.github.io}{gleaned}} --- \emph{Author.} {\footnotesize Collection of my book summaries. Status: Pilot.}\\
\textsc{\href{https://donkeydocs.github.io}{donkeyDocs}} --- \emph{Contributor.} {\footnotesize A repository of lecture notes. Status: Initialising. }\\

\newpage
\par{\centering\Large \hypertarget{grds}{Bachelor and Master of Science with a major in \textsc{Physics}}\par}\normalsize
\setlength{\extrarowheight}{7pt}

\begin{center}
\label{grades}
\begin{tabular}{cp{9cm}c}
  \textsc{Semester$^*$} & \multicolumn{1}{c}{\textsc{Subjects}} & \textsc{Score} \\
  \hline
  1 & Mechanics, Chemistry of elements and chemical transformations, 
  Cellular basis of life, Symmetry, Language skills B (English), 
  Introduction to computers, Physics lab I, Chemistry lab I, Biology lab I &8.5/10 \\
  2 & Electromagnetism, Atoms molecules and symmetry, Gene expression and development, Analysis in one variable,
  Hands-on electronics, History of science, Physics lab II, Chemistry lab II, Biology lab II &8.6/10 \\
  3 & Waves and optics, Spectroscopic and other physical methods, Genetics and evolution, Curves and surfaces, 
  Introduction to astrophysics, Workshop training, Physics lab III, Chemistry lab III, Biology lab III &8.8/10\\
  4 & Thermodynamics and statistical physics, Energetics and dynamics of chemical reactions, Behaviour and ecology,
  Probability and statistics, Introduction to quantum physics, Philosophy of science, Physics lab IV, Chemistry lab IV,
  Biology lab IV& 9.7/10 \\
  5$^\dagger$ & Classical mechanics, Quantum mechanics, Electrodynamics, Advanced optics lab, Reason and rationality & 10/10\\
  6 & Statistical mechanics, Atomic and molecular physics, Quantum computation, Advanced electronics and instrumentation
  lab, Quantum field theory & 9.6/10 \\
  7 & Solid state physics, Nuclear and particle physics, Nuclear physics lab, Physics of fluids, Quantum principles 
  and quantum optics, Radiative effects and renormalisation group in relativistic quantum field theory & 9.4/10\\
  8 & Nonlinear dynamics, Chaos and complex systems, Condensed matter physics lab, computational methods in physics,
  Standard model and beyond, Selected topics in classical and quantum mechanics& 9.5/10\\
  9 & Ethics, MS Thesis---Research project I & 10/10 \\
  10 & Cosmology and galaxy formation, MS Thesis---Research project II & 10/10\\
  & \cline{2-2}
  &Cumulative Performance Index (CPI)& \textbf{9.4} /10\\
  & \\
  \hline
\end{tabular}
\end{center}
\vspace*{\fill}
\hrule
\footnotesize{$*$ Note that the credits associated with each semester are not exactly the same.}

\footnotesize{$\dagger$ Physics major henceforth.}

%\newpage
%\hypertarget{gmat}{\textsc{Gmat}\setmainfont{LMRoman10 Regular}\textregistered\setmainfont[SmallCapsFont=Fontin-SmallCaps]{Fontin-Regular}}

%\XeTeXpdffile ''GMAT.pdf'' page 1 scaled 800

\end{document}
