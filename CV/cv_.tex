\documentclass[a4paper,10pt]{article}

%A Few Useful Packages
\usepackage{graphicx}
\usepackage[absolute,overlay]{textpos}
  \setlength{\TPHorizModule}{1cm}
  \setlength{\TPVertModule}{1cm}

  \usepackage{longtable}

\usepackage{marvosym}
\usepackage{fontspec} 					%for loading fonts
\usepackage{xunicode,xltxtra,url,parskip} 	%other packages for formatting
\RequirePackage{color,graphicx}
\usepackage[usenames,dvipsnames]{xcolor}
\usepackage[big]{layaureo} 				%better formatting of the A4 page
% an alternative to Layaureo can be ** \usepackage{fullpage} **
\usepackage{supertabular} 				%for Grades
\usepackage{titlesec}					%custom \section
\usepackage{array}
%Setup hyperref package, and colours for links
\usepackage{hyperref}
\definecolor{linkcolour}{rgb}{0,0.2,0.6}
\hypersetup{colorlinks,breaklinks,urlcolor=linkcolour, linkcolor=linkcolour}

%FONTS
\defaultfontfeatures{Mapping=tex-text}
%\setmainfont[SmallCapsFont = Fontin SmallCaps]{Fontin}
%%% modified for Karol Kozioł for ShareLaTeX use
\setmainfont[
SmallCapsFont = Fontin-SmallCaps.otf,
BoldFont = Fontin-Bold.otf,
ItalicFont = Fontin-Italic.otf
]
{Fontin.otf}
%%%%gla

%CV Sections inspired by: 
%http://stefano.italians.nl/archives/26
%\titleformat{\section}{\Large\scshape\raggedright}{}{0em}{}[\titlerule]
\titleformat{\section}{\Large\scshape\raggedright}{}{0em}{}%[\titlerule]
\titlespacing{\section}{0pt}{15pt}{5pt}
%Tweak a bit the top margin
%\addtolength{\voffset}{-1.3cm}

%Italian hyphenation for the word: ''corporations''
\hyphenation{im-pre-se}

%-------------Custom commands----------------------
\newcommand{\su}[1]{{\tiny $^#1$}}

%-------------WATERMARK TEST [**not part of a CV**]---------------
\usepackage[absolute]{textpos}

\setlength{\TPHorizModule}{30mm}
\setlength{\TPVertModule}{\TPHorizModule}
\textblockorigin{2mm}{0.65\paperheight}
\setlength{\parindent}{0pt}

%--------------------BEGIN DOCUMENT----------------------
\begin{document}

%WATERMARK TEST [**not part of a CV**]---------------
%\font\wm=''Baskerville:color=787878'' at 8pt
%\font\wmweb=''Baskerville:color=FF1493'' at 8pt
%{\wm 
%	\begin{textblock}{1}(0,0)
%		\rotatebox{-90}{\parbox{500mm}{
%			Typeset by Alessandro Plasmati with \XeTeX\  \today\ for 
%			{\wmweb \href{http://www.aleplasmati.comuv.com}{aleplasmati.comuv.com}}
%		}
%	}
%	\end{textblock}
%}

\pagestyle{empty} % non-numbered pages

\font\fb=''[cmr10]'' %for use with \LaTeX command

%--------------------TITLE-------------
%\flushright
%\centering
\par{  
		{\Huge Atul Singh \textsc{Arora}
	}  \bigskip\par}
\bigskip\bigskip\bigskip
%\begin{textblock}{10}(4.7,-5.9) %right aligned
\begin{textblock}{10}(4.6,-5.6) %right aligned
%\begin{textblock}{10}(1.1,-5.9) %left aligned
%\begin{textblock}{10}(3,-5.9) %middle
     \includegraphics[width=2.2cm]{atul.jpg}
    \end{textblock}

%--------------------SECTIONS-----------------------------------
%Section: Personal Data
\section{Personal}

\begin{tabular}{rl}
    %\textsc{Place and Date of Birth:} & Someplace, Italy  | dd Month 1912 \\
    \textsc{Address:}   & 1318 Cordova St, Pasadena, CA 91106, USA \\
    \textsc{Phone:}     & +1 626 318 0732\\ 
    %\textsc{Mobile:}      
                        & +1 626 515 4073\\
    \textsc{Email:}     & \href{mailto:asarora@caltech.edu}{asarora@caltech.edu}, \href{mailto:atul.singh.arora@gmail.com}{atul.singh.arora@gmail.com}%\href{https://atulsingharora.github.io}{atulsingharora.github.io}
\end{tabular}

%Section: Work Experience at the top
\section{Research}

\begin{longtable}{r|p{11cm}}
  \textsc{2021-}\emph{present}
                   & PostDoc, \textsc{California Institute of Technology, United States} \\
                   &\small Advisor: Prof. Thomas \textsc{Vidick}\\
                    &\footnotesize{Showed oracle separations of hybrid quantum-classical circuits and recovered previous results (original proofs had errors).\su{1}} \\
                    &\footnotesize{Constructed a more robust proof of quantumness, based on computational assumptions. It uses only 2 rounds and does not require the adaptive hardcore bit property (e.g. the Rabin cryptosystem suffices).\su{2}} \\
                    %&\footnotesize{Completed older projects:}\\
                    &\footnotesize{Motivated by contextuality, demonstrated self-testing of a single quantum system (includes both theory and experiment).\su{3}} \\
                    &\footnotesize{Introduced methods to improve the security of device-independent weak coin flipping protocols, resulting in an improvement after a decade.\su{4}}\\ 
                    &\footnotesize{Collected all our previous results on the topic into a journal version---Solutions to Quantum Weak Coin Flipping.\su{5}}\\

                    &\small{{\tiny $^1$} ASA, A. Gheorghiu, U. Singh. \href{https://arxiv.org/abs/2201.01904}{arXiv:2201.01904} (submitted; \href{https://atulsingharora.github.io/HQC}{web})}\\                                                                     
                    &\small{{\tiny $^2$} K. Bharti, S. Hung, ASA, et. al. (in preparation. \href{https://www.overleaf.com/read/qzxbvnsfnvxp}{overleaf}}) \\
                    &\small{{\tiny $^3$} X. Hu, Y. Xie, ASA, M. Ai, K. Bharti, et. al. \href{https://arxiv.org/abs/2203.09003}{arXiv:2203.09003} (submitting)}\\  
                    &\small{{\tiny $^4$} ASA, J. Sikora, T Van Himbeeck (submitting; \href{http://atulsingharora.github.io/DI_WCF}{web})}\\
                    &\small{{\tiny $^5$} ASA, J. Roland, C. Vlachou, S. Weis. (submitted; \href{https://www.overleaf.com/read/khztgjgdgvmm}{overleaf}})\\
  \multicolumn{2}{c}{} \\                   
 \textsc{2016-20}     & PhD Thesis, \textsc{Université libre de Bruxelles (ULB)}, Belgium \\
                    &\emph{Quantum Weak Coin Flipping} \\
                    &\small Advisor: Prof. Jérémie \textsc{Roland}\\
                    &\footnotesize{Primarily working on quantum weak coin flipping, a cryptographic primitive. Its figure of merit is called the bias, $\epsilon$. The best known had $\epsilon \to 1/6$ by C. Mochon in 2005.} \\
                    &\footnotesize{~~End 2017: Protocols with $\epsilon \to 1/10$ were found{\tiny $^1$}. }\\
                    &\footnotesize{~~End 2018: An algorithm to numerically find protocols with $\epsilon\to 0$ was given{\tiny $^1$}. }\\
                    &\footnotesize{~~End 2019: An exact (geometric) solution to the problem was found{\tiny $^2$}. }\\
                    &\footnotesize{~~Mid 2020: A simpler, exact (algebraic) solution to the problem was found{\tiny $^3$}. }\\
                    &\footnotesize{On the side, investigated foundational aspects of quantum mechanics{\tiny $^4$}. }\\
                    %&\footnotesize{Current: Simplifying the exact solution. Exploring self-testing using contextuality. Formulating continuous time quantum communication complexity.}\\
                    &\small{{\tiny $^1$}ASA, J. Roland, S. Weis. \href{https://arxiv.org/abs/1811.02984}{arXiv:1811.02984} (\href{https://www.youtube.com/watch?v=eNK6X7BlG5U&list=PLGdMsPGuoD25wLgnY7RBoTAxsnQEMsNA0&index=12}{QIP '19} \href{http://dx.doi.org/10.1145/3313276.3316306}{STOC '19} \href{https://atulsingharora.github.io/WCF}{web}) }\\
                    &\small{{\tiny $^2$}ASA, J. Roland, C. Vlachou. \href{https://arxiv.org/abs/1911.13283v1}{arXiv:1911.13283v1}} (\href{https://atulsingharora.github.io/WCF_2}{web})\\
                    &\small{{\tiny $^3$}ASA, J. Roland, C. Vlachou. \href{https://arxiv.org/abs/1911.13283v2}{arXiv:1911.13283v2} (\href{https://youtu.be/A2GRxspzWUg?t=801}{QCrypt '20} \href{https://youtu.be/nlZ5JhoE0D8}{QIP '21} \href{https://doi.org/10.1137/1.9781611976465.58}{SODA '21} \href{https://atulsingharora.github.io/WCF_2}{web}})\\
                    &\small{{\tiny $^4$}K. Bharti, A.S.A, L. C. Kwek, J. Roland. \href{https://arxiv.org/abs/1811.05294}{arXiv:1811.05294} (\href{https://link.aps.org/doi/10.1103/PhysRevResearch.2.033010}{Phys. Rev. Res. 2, 033010}) }\\
 \multicolumn{2}{c}{} \\

 \textsc{2015-16} & Master's Thesis, \textsc{Indian Institute of Science Education and Research (IISER), Mohali}, India \\
                  &\emph{Contextuality in a Deterministic Quantum Theory}\\
                  &\small Advisor: Prof. Arvind\\ 
                  &\footnotesize{Concluded that contextuality is not a necessary feature of quantum mechanics and proposed an alternative, non functional-consistency, bolstered by an explicit construction.}\\
                  &\small{ASA, K. Bharti, Arvind. \href{https://arxiv.org/abs/1607.03498}{arXiv:1607.03498}; \href{https://doi.org/10.1016/j.physleta.2018.11.049}{Physics Letters A. (Nov 2018)} }\\            
\multicolumn{2}{c}{} \\

\textsc{Summer}   & Internship \textsc{University of Siegen}, Germany\\
2015                  & \emph{Towards a macroscopic test of local realism}\\
                      & \small{Advisor: Prof. Otfried \textsc{Gühne}}\\
                      & \footnotesize{Constructed a Bell inequality using observables bounded in phase space to probe local realism using macroscopic variables.} \\
                      & \small{ASA, A. Asadian. \href{https://arxiv.org/abs/1508.04588}{arXiv:1508.04588}; \href{http://dx.doi.org/10.1103/PhysRevA.92.062107}{Phys. Rev. A 92, 061207} }\\
                      
\multicolumn{2}{c}{} \\

\textsc{2011-14}  & Internships \\
                      &\textsc{IISER Mohali}, India. {\footnotesize Quantum simulation (theory).} 
                      {\footnotesize Advisor:} \small{Prof Arvind}.\\
                      &\textsc{National Physical Laboratory (NPL)}, New Delhi, India. {\footnotesize Set up an experiment to study the dynamics of a dipole lattice.}                    
                      {\footnotesize Advisor:} \small{Dr Ravi \textsc{Mehrotra}}.\\
                      &\textsc{Indian Institute of Technology (IIT), Bombay, India.} {\footnotesize Yarn defect recognition using OpenCV.}  
                      {\footnotesize Advisor:} \small{Prof Anirban \textsc{Guha}}.
\end{longtable}

%Section: Education
\section{Education}
\begin{tabular}{rp{11cm}}	
  \textsc{Sep} 2020 & Doctorat en Sciences de l'ingénieur et technologie,\\
 \textsc{Oct} 2016  & \textbf{Université libre de Bruxelles (ULB)}, Belgium.\\
                    & \\
 \textsc{July} 2016 & Bachelor and Master of Science with \textsc{Physics} major,\\
 \textsc{July} 2011 & \textbf{Indian Institute of Science Education and Research (IISER), Mohali}, India.\\
& CPI: \textbf{9.4} /10. \small Graduated with \emph{rank two.} \hyperlink{grades}{\hfill | \footnotesize Details at the end}\\&\\ 
% & Thesis: ``Sublinear and Locally Sublinear Prices'' | \small Advisor: Prof. Erio \textsc{Castagnoli}\\
% &\normalsize \textsc{Gpa}: 28.61/30\hyperlink{grds}{\hfill | \footnotesize Detailed List of Exams}\\&\\
\end{tabular}

\section{Conferences}
\begin{longtable}{rrp{11cm}}
  & ~~2022 &\textbf{Poster}. Oracle separations of hybrid quantum-classical circuits\\
  &         &Quantum Information Processing (\textsc{QIP}). Caltech, USA\\
  & ~~2022 &\textbf{Poster}. Improving the security of device independent weak coin flipping protocols.\\
  &         &Quantum Information Processing (\textsc{QIP}). Caltech, USA\\  
  & ~~2021 &\textbf{Talk}. \emph{Analytic quantum weak coin flipping protocols with arbitrarily small bias}. \\
  &         &ACM-SIAM Symposium on Discrete Algorithms (\textsc{SODA}). Virtual\\    
  & ~~2021 &\textbf{Talk}. \emph{Analytic quantum weak coin flipping protocols $\dots$}. \\
  &         &Quantum Information Processing (\textsc{QIP}). Virtual/Munich, Germany.\\  
  & ~~2020 &\textbf{Talk}. \emph{Analytic quantum weak coin flipping protocols $\dots$}. \\
  &         &\textsc{QCrypt}. Virtual/Amsterdam, Netherlands.\\
  & ~~2019  &\textbf{Participant}. \\
  & ~~~~~~ &\textsc{QuantAlgo} Workshop. CWI, Amsterdam, Netherlands.\\
  & ~~2019 &\textbf{Participant}. \\
  & ~~~~~~ &(Physics) Lindau Nobel Laureate Meeting (\textsc{LiNo}). Lindau, Germany. \\
  & ~~2019 &\textbf{Talk}. \emph{Quantum Weak Coin Flipping}. \\
  & ~~~~~~ &Symposium on Theory of Computing (\textsc{STOC}). Phoenix, Arizona, USA. \\
  & ~~2019 &\textbf{Talk}. \emph{Quantum Weak Coin Flipping}. \\
  & ~~~~~~ &Quantum Information Processing (\textsc{QIP}). University of Colorado, USA. \\
  & ~~2018 &\textbf{Talk}. \emph{Quantum Weak Coin Flipping beyond bias 1/6}. \\  
  & ~~~~~~ &\textsc{QuantAlgo} Workshop. Université Paris-Diderot, Paris, France. \\
  & ~~2018 &\textbf{Poster}. \emph{Quantum Weak Coin Flipping with bias 1/10}. \\
  & ~~~~~~ &Quantum Information Processing (\textsc{QIP}). TU Delft, Netherlands. \\
  & ~~2017 &\textbf{Participant}. \\
  & ~~~~~~ &Theory of Quantum Computation, Communication and Cryptography (\textsc{TQC}). Paris, France.
  \end{longtable}

%Section: Scholarships and additional info
\section{Recognition }
\begin{tabular}{rrp{11cm}}
  & ~~2019 & Granted financial support for attending the \emph{(Physics) Lindau Nobel Laureate Meeting, 2019}. \\
  & ~~2018 & Renewed. Two year research fellowship from the Belgian \emph{Fonds National Recherche de Science (FNRS)}, through the FRIA grant.\normalsize\\
  & ~~2016 & Awarded. Two year research fellowship from the Belgian \emph{Fonds National Recherche de Science (FNRS)}, through the FRIA grant.\normalsize\\
 & ~~2016     & Top 5\% in the physics stream of the \emph{Graduate Aptitude Test in Engineering (GATE)}, India. \\
 & ~~~~~~     & Obtained a 92.3 percentile in the national graduate physics exam, \emph{Joint Entrance Screening Test (JEST)}, India. \\
% \end{tabular}

% \begin{tabular}{rp{11cm}}
 & ~~2015     & Awarded the \emph{Junior Research Fellowship (JRF-NET)} from the Council of Scientific and Industrial Research, India. \\
 & ~~~~~~     & Awarded the \emph{DAAD WISE} fellowship for a summer internship by and in Germany.\\
 & 2013-16  & Awarded the Certificate of Merit for the best academic performance in a semester, twice by IISER. Was among the highest scorers four other times.\\
 & ~~2012     & Awarded the \emph{KVPY} fellowship for my work on Stepper Motor Control, by DST, India.\\
 & ~~2010     & Granted financial support for attending the Bright Green Youth climate summit, Denmark.
\end{tabular}


  \section{Teaching}
  \begin{tabular}{rrp{11cm}}
  & ~~2019 &Teaching Assistant. Information Quantique (graduate). ULB, Brussels.\\  
  & ~~2016 &Teaching Assistant. Thermodynamics (undergraduate). IISER, Mohali.\\
  & ~~2015 &Teaching Assistant. Classical Mechanics (undergraduate). IISER, Mohali.
  \end{tabular}
  

%Section: Languages
\section{Languages}
\begin{tabular}{rl}
\textsc{English:}&Fluent\\
\textsc{French:}&Basic\\
\textsc{Hindi:}&Fluent\\
\textsc{Punjabi:}&Intermediate\\
\end{tabular}


% \section{Computer Skills}
% \begin{tabular}{rl}
%  Basic Knowledge:& \textsc{php}, my\textsc{sql}, \textsc{html}, Access, \textsc{Linux}, ubuntu, {\fb \LaTeX}\setmainfont[SmallCapsFont=Fontin-SmallCaps.otf]{Fontin.otf}\\
% Intermediate Knowledge:& \textsc{vba}, Excel, Word, PowerPoint\\
% \end{tabular}

\section{Interests \& Extracurricular}
Technology, Open-Source, Programming;\\ 
Philosophy, Reading; \\
Fitness; Piano, Guitar, Violin.
% \section{Extracurricular}

% \textsc{\href{https://qlic-meets.github.io}{QLIC-meets}} --- \emph{Organizer.} {\footnotesize Department meetings/lectures held to facilitate collaboration. Status: Running. } \\
% \textsc{\href{https://c-est-ca.github.io}{C'est ça}} --- \emph{Editor.} {\footnotesize An at least bilingual peer-reviewed popular science journal. Status: Initialising. }\\
% \textsc{\href{https://gleaned.github.io}{gleaned}} --- \emph{Author.} {\footnotesize Collection of my book summaries. Status: Pilot.}\\
%\textsc{\href{https://donkeydocs.github.io}{donkeyDocs}} --- \emph{Contributor.} {\footnotesize A repository of lecture notes. Status: Initialising. }\\

\newpage
\par{\centering\Large \hypertarget{grds}{Bachelor and Master of Science with a major in \textsc{Physics}}\par}\normalsize
\setlength{\extrarowheight}{7pt}

\begin{center}
\label{grades}
\begin{tabular}{cp{9cm}c}
  \textsc{Semester$^*$} & \multicolumn{1}{c}{\textsc{Subjects}} & \textsc{Score} \\
  \hline
  1 & Mechanics, Chemistry of elements and chemical transformations, 
  Cellular basis of life, Symmetry, Language skills B (English), 
  Introduction to computers, Physics lab I, Chemistry lab I, Biology lab I &8.5/10 \\
  2 & Electromagnetism, Atoms molecules and symmetry, Gene expression and development, Analysis in one variable,
  Hands-on electronics, History of science, Physics lab II, Chemistry lab II, Biology lab II &8.6/10 \\
  3 & Waves and optics, Spectroscopic and other physical methods, Genetics and evolution, Curves and surfaces, 
  Introduction to astrophysics, Workshop training, Physics lab III, Chemistry lab III, Biology lab III &8.8/10\\
  4 & Thermodynamics and statistical physics, Energetics and dynamics of chemical reactions, Behaviour and ecology,
  Probability and statistics, Introduction to quantum physics, Philosophy of science, Physics lab IV, Chemistry lab IV,
  Biology lab IV& 9.7/10 \\
  5$^\dagger$ & Classical mechanics, Quantum mechanics, Electrodynamics, Advanced optics lab, Reason and rationality & 10/10\\
  6 & Statistical mechanics, Atomic and molecular physics, Quantum computation, Advanced electronics and instrumentation
  lab, Quantum field theory & 9.6/10 \\
  7 & Solid state physics, Nuclear and particle physics, Nuclear physics lab, Physics of fluids, Quantum principles 
  and quantum optics, Radiative effects and renormalisation group in relativistic quantum field theory & 9.4/10\\
  8 & Nonlinear dynamics, Chaos and complex systems, Condensed matter physics lab, computational methods in physics,
  Standard model and beyond, Selected topics in classical and quantum mechanics& 9.5/10\\
  9 & Ethics, MS Thesis---Research project I & 10/10 \\
  10 & Cosmology and galaxy formation, MS Thesis---Research project II & 10/10\\
  & \cline{2-2}
  &Cumulative Performance Index (CPI)& \textbf{9.4} /10\\
  & \\
  \hline
\end{tabular}
\end{center}
\vspace*{\fill}
\hrule
\footnotesize{$*$ Note that the credits associated with each semester are not exactly the same.}

\footnotesize{$\dagger$ Physics major henceforth.}

%\newpage
%\hypertarget{gmat}{\textsc{Gmat}\setmainfont{LMRoman10 Regular}\textregistered\setmainfont[SmallCapsFont=Fontin-SmallCaps]{Fontin-Regular}}

%\XeTeXpdffile ''GMAT.pdf'' page 1 scaled 800

\end{document}
